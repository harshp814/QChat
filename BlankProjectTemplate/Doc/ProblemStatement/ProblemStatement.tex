\documentclass{article}


\title{SE 3XA3: Problem Statement \\ \textbf{QChat - QuestionChat}}

\author{Group 14
		\\\textbf{Adit Patel}     - Patela14
		\\\textbf{Harsh Patel}    - Patelh11
		\\\textbf{Vrushesh Patel} - Patelv12
}

\date{September 25, 2017}

\usepackage{fullpage}


\begin{document}

\maketitle
\newpage

\noindent\textbf{What problem are you trying to solve?}\\
\\For various different reasons students are often unable to voice concerns and questions
during class/lecture time, we are trying to solve this by creating a platform where they
can do so. The problem is that some students are shy, have anxiety disorders or
sometimes they do not get a chance to ask questions in class because they do not want
to disturb the lecture. Other times the teacher/professor is on a tight deadline and does
not want to take questions until the very end, but by this time the student has already
either forgotten the question or missed too many connecting concepts and is entirely
lost.
\\
\\
\\
\noindent\textbf{Why is this an important problem?}\\\\
This is an important issue to tackle because it hinders the students learning experience.
To elaborate, all students are capable of learning and how they do so may differ,
however they are expected to go to lecture and they should be making the most out of
the time they have with the professor present. When students do not voice their
questions or are not given the opportunity, they miss out on clarifications that could be
sometimes very simple. This results in students dedicating too much time solving simple
problems that have already been discussed in lecture.
\\
\\
\\
\noindent\textbf{What is the context of the problem you are solving?}\\\\
Redesigning the software to a web application so that users don’t have to download any
additional programs on their machine, making it more accessible to everyone. Also our
solution will be an update to the generic chat room to a question and answers platform
focused on real time communication.
\\
\\
\\
\noindent\textbf{Who are the stakeholders?}\\\\
As our problem is an issue that occurs in the university ecosystem, most of our
stakeholders will be members of this ecosystem. Our stakeholders are students and
professors. Students will be impacted heavily as they will move to a friendly platform to
ask questions to colleagues and/or the professor. As students use this platform to
interact in a new way, professors will be impacted heavily as well. Being expected to
resolve questions through/from this platform may need them to reform the way they
deliver lectures to allocate enough time to review and answer questions from the
platform.
\\
\\
\\
\noindent\textbf{What is the environment for the software?}\\\\
Since the application is going to run on a web browser, the environment that the
software can run on includes desktops/laptops browsers, browsers on cellphones etc.
Any devices that has access to internet and a web browser.


\end{document}